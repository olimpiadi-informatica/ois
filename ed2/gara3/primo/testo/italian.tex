\usepackage{xcolor}
\usepackage{afterpage}
\usepackage{pifont,mdframed}
\usepackage[bottom]{footmisc}

\makeatletter
\gdef\this@inputfilename{input.txt}
\gdef\this@outputfilename{output.txt}
\makeatother

\newcommand{\inputfile}{\texttt{input.txt}}
\newcommand{\outputfile}{\texttt{output.txt}}

\newenvironment{warning}
  {\par\begin{mdframed}[linewidth=2pt,linecolor=gray]%
    \begin{list}{}{\leftmargin=1cm
                   \labelwidth=\leftmargin}\item[\Large\ding{43}]}
  {\end{list}\end{mdframed}\par}

William e Giorgio stanno facendo un gioco con i cosiddetti numeri \emph{primi permissivi}. Un numero primo permissivo è un numero in cui alcune delle cifre possono essere degli asterischi. Un esempio di numero primo permissivo è $3\star$ il quale, in base al valore che sostituiremo all'asterisco, potrebbe valere $31$ oppure $37$, che sono entrambi dei numeri primi. Nessun numero primo permissivo comincia per asterisco.

Aiuta i protagonisti a contare, dato un certo numero primo permissivo, a quanti numeri primi esso può corrispondere.

\Implementation
Dovrai sottoporre esattamente un file con estensione \texttt{.c}, \texttt{.cpp} o \texttt{.pas}.

\begin{warning}
Tra gli allegati a questo task troverai un template (\texttt{primo.c}, \texttt{primo.cpp}, \texttt{primo.pas}) con un esempio di implementazione da completare.
\end{warning}

Se sceglierai di utilizzare il template, dovrai implementare la seguente funzione:
\begin{center}\begin{tabularx}{\textwidth}{|c|X|}
\hline
C/C++  & \verb|int primi(char S[]);|\\
\hline
Pascal & \verb|function primi(var S: string): longint;|\\
\hline
\end{tabularx}\end{center}
In cui:
\begin{itemize}[nolistsep]
  \item $S$ è una stringa che rappresenta il numero primo permissivo scelto. Per calcolarne la lunghezza si può usare la funzione \texttt{strlen} in C/C++ o la funzione \texttt{length} in Pascal.
  \item La funzione dovrà restituire il totale di numeri primi che possono corrispondere al numero primo pemissivo dato in input. Questo totale verrà stampato sul file di output.
\end{itemize}

\InputFile
Il file \inputfile{} è composto da una sola riga che contiene la stringa $S$.

\OutputFile
Il file \outputfile{} è composto da un'unica riga contenente un unico intero, la risposta a questo problema.

% Assunzioni
\Constraints
\begin{itemize}[nolistsep, itemsep=2mm]
  \item $2 \le |S| \le 7$, dove $|S|$ è la lunghezza di $S$.
  \item Il primo carattere non è un asterisco.
  \item C'è sempre almeno un asterisco.
\end{itemize}

\Scoring
Il tuo programma verrà testato su diversi test case raggruppati in subtask.
Per ottenere il punteggio relativo ad un subtask, è necessario risolvere
correttamente tutti i test relativi ad esso.

\begin{itemize}[nolistsep,itemsep=2mm]
  \item \textbf{\makebox[2cm][l]{Subtask 1} [10 punti]}: Casi d'esempio.
  \item \textbf{\makebox[2cm][l]{Subtask 2} [30 punti]}: $|S| \leq 3$.
  \item \textbf{\makebox[2cm][l]{Subtask 3} [30 punti]}: $|S| \leq 5$.
  \item \textbf{\makebox[2cm][l]{Subtask 4} [30 punti]}: Nessuna limitazione specifica.
\end{itemize}

% Esempi


\Examples
\begin{example}
\exmpfile{primo.input0.txt}{primo.output0.txt}%
\exmpfile{primo.input1.txt}{primo.output1.txt}%
\exmpfile{primo.input2.txt}{primo.output2.txt}%
\end{example}


\Explanation
Il \textbf{primo caso di esempio} corrisponde a quello nella descrizione del problema.\\[2mm]
Nel \textbf{secondo caso di esempio} i seguenti primi vanno bene: $103, 113, 163, 173, 193$.
