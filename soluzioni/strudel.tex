% autore: Nicolò Mazzucato

 \createsection{\N2Lenta}{{\small{$\blacksquare$}}                             \normalsize Una soluzione $O(N^2)$ }
 
  \createsection{\NlogN}{{\small{$\blacksquare$}}                             \normalsize Una soluzione $O(NlogN)$ }
   \createsection{\NLineare}{{\small{$\blacksquare$}}                             \normalsize Una soluzione $O(N)$ }
 \N2Lenta
 Una soluzione molto intuitiva può essere quella di partire da ogni punto dello strudel e prendere \textit{1, 2, 3, ...} fette consecutive, verificando ogni volta che la quantità di cannella sia almeno uguale alla quantità di mandorle nella porzione. Fatto questo si tiene conto della fetta di dimensione più grande.
Si possono effettuare diverse ottimizzazioni a questo metodo, riuscendo comunque a superare al massimo i primi 3 testcase.

\NlogN
Per semplicità d'ora in avanti sostituiremo i due array riguardanti la quantità di cannella e mandorle in ogni fetta con un terzo, che chiameremo \textit{bonta} dove ad ogni posizione \textit{i} corrisponderà \textit{cannella[i]-mandorle[i]}.
Il problema è ora ridotto al trovare una sequenza continua di valori all'interno del nostro array \textit{bonta} in cui la somma dei singoli elementi sia maggiore di zero.
Per far questo possiamo aiutarci con le somme cumulate:
creiamo un nuovo vettore \textit{sum} di dimensioni N, e lo riempiamo in modo tale che valga la proprietà:     \[sum[k] =  \sum_{i=0}^k bonta[i]\]

Ora per trovare la somma degli elementi in un intervallo delimitato dagli indici \textit{i} e \textit{j con i $\le$ j} (escluso i)  basterà fare \textit{sum[j] - sum[i]}, ovvero sottrarre  alla somma di tutti gli elementi fino a \textit{j} (fine della sequenza), la somma di tutti gli elementi antecedenti a \textit{i} (inizio della sequenza).

Ora possiamo notare facilmente che  la somma dei valori dell'array originario \textit{bonta} tra due indici \textit{i} e \textit{j con i $\le$ j} sarà 0 quando i due valori nel corrispettivo array \textit{sum} saranno uguali:
 \[sum [j] = sum[i] \rightarrow sum[j] - sum[i] = 0\]
Sarà maggiore di 0 quanto all'indice \textit{i}  di \textit{sum} c'è un valore minore che all'indice \textit{j}.
Sfruttando questra prorietà riduciamo il problema al trovare una coppia di indici la cui distanza sia la maggiore (in quanto viene chiesto di trovare la fetta più lunga) e la loro differenza sia maggiore o uguale a 0 ( la quantità di cannella dev'essere almeno uguale alla quantità di mandorle).
Un modo per trovare la risposta è considerare ogni coppia di indici, analogamente alla prima soluzione $O(N^2)$, procedimento troppo dispendioso in termini di tempo.
Possiamo notare un'altra particolarità.
Possiamo ridurre i possibili punti di partenza trovando dei candidati punti di partenza, attraverso i "minimi parziali". 
Creiamo un vettore chiamato \textit{minimi\_parziali} e mettiamoci il primo elemento di \textit{sum}. Ora iteriamo su tutti i restanti elementi di \textit{sum}, inserendoli in \textit{minimi\_parziali} solo se minori dell'ultimo elemento contenuto.
Perchè fare questo? Perchè in questo modo avremo una lista ordinata in modo decrescente di candidati punti di partenza; è semplice notare che non avrebbe senso considerare punti di inizio all'infuori di quelli nel vettore \textit{minimi\_parziali}, in quanto questi soddisfano la condizione di essere i minori in un dato intervallo. Se si dovesse scegliere un altro punto \textit{x} esterno alla lista, sarebbe facile dimostrare che prendere il primo punto in \textit{minimi\_parziali} con indice minore di \textit{x} migliorerebbe la soluzione, in quanto renderebbe l'intervallo più lungo, mantenendo un valore minore o uguale ad \textit{x}.
Ora per ogni indice \textit{i} di \textit{sum} dovremo trovare il valore contenuto in \textit{minimi\_parziali} che sottratto a \textit{sum[i]} dia un valore maggiore o uguale a 0, più distante da \textit{i} . Per far questo possiamo avvalerci di una ricerca dicotomica all'interno dell'array \textit{minimi\_parziali}.
Descrivendo brevemente il procedimento, dobbiamo trovare l'indice \textit{p} di partenza per l'indice \textit{d} di destinazione (indice che itererà su tutti gli elementi).
Inizieremo stimando che il valore ricercato sia al centro dell'array \textit{minimi\_parziali}, e verificheremo la condizione che caratterizza la soluzione, ovvero se il valore \textit{sum[d]} $\ge$ \textit{sum[p]}. In caso affermativo avremo una possibile soluzione, ciò che dobbiamo verificare è se questa è migliorabile, attraverso lo stesso procedimento, cambiando l'intervallo di ricerca.

\NLineare
Prendendo spunto dalla soluzione precedente, ci si può rendere conto che si può effettuare lo stesso procedimento fatto con i punti di partenza, con i punti di destinazinoe della soluzione, con la differenza di trovare i massimi parziali, iniziando però dalla fine dell'array \textit{sum}.
una volta fatto questo si avranno due vettori, con tutti i candidati punti di partenza e arrivo.
Entrami saranno ordinati: i minimi saranno in ordine decrescente, e i massimi in ordine crescente.
Ora si può effettuare un procedimento che può ricordare il \textit{merge} tra i due array, spostando semplicemente due indici, e tenendo conto della soluzione migliore.

\colorbox{white}{\makebox[.99\textwidth][l]{\includegraphics[scale=.73]{codice.pdf}}}
