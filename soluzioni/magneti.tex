% autore: Giulio Carlassare

\createsection{\Codice}{{\small{$\blacksquare$}} \normalsize Codice C++}
\createsection{\SolN}{{\small{$\blacksquare$}} \normalsize Soluzione $O(n)$}

Il problema chiede di fare in modo che, al termine dell'elaborazione, tutti i magneti siano ruotati nello \textbf{stesso verso}.

\SolN

Si può notare che sarà necessario girare \textbf{tutti e soli} i magneti che sono inizialmente posizionati in un certo verso.\newline
Ad esempio si potranno girare tutti i $\mpm$ e farli diventare $\mmp$.\newline
Dato che è richiesto il numero minimo di rotazioni, si dovrà contare il numero di magneti $\mpm$ e il numero di $\mmp$ e decidere ovviamente qual è il minimo.

Si nota facilmente che avendo in tutto $N$ magneti, e $M$ magneti $\mpm$, quelli $\mmp$ saranno $N-M$.

Questo algoritmo risolve il problema con complessità $O(N)$ ed è quello ottimo.

\Codice

\includegraphics{code.pdf}