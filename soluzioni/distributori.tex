% autore: Giulio Carlassare

\createsection{\Codice}{{\small{$\blacksquare$}} \normalsize Codice C++}
\createsection{\SolN}{{\small{$\blacksquare$}} \normalsize Soluzione $O(n)$}

\SolN

Il problema può essere risolto \textit{''simulando''} il viaggio di Gabriele, avanzando fino a quando è necessario fare benzina.


Si analizzino i distributori in ordine, controllando per ognuno se quello sucessivo rientra ancora nella distanza percorribile e può quindi essere raggiunto.\newline
In caso positivo è possibile continuare a viaggiare; in caso contrario bisogna \textbf{fermarsi al distributore corrente}, perchè altrimenti la benzina finirebbe nel tragitto fino al prossimo.\newline
Una volta effettuato il pieno si riparte con lo stesso metodo, ma considerando la posizione a cui ci siamo fermati come se fosse l'inizio del viaggio.\newline
Un'eccezione si incontra all'ultimo distributore: è necessario controllare se è possibile arrivare alla fine del viaggio senza fermarsi. In caso negativo bisogna fermarsi per fare benzina.\newline
Alla fine le fermate ai distributori saranno il numero minimo per arrivare alla distanza prestabilita.

\textbf{Secondo caso di prova:}

\begin{verbatim}
Distanza massima con un pieno:	30
Distanza da raggiungere:				100

Distributori                     Inizio	  Fermate   Commenti

1 31 33 38 62 69 93 97 98 99     0        0         Inizio del viaggio

1 31 33 38 62 69 93 97 98 99     1        1         Bisogna fermarsi
^                                                   (non si riesce ad arrivare al prossimo)
1 31 33 38 62 69 93 97 98 99     31       2         Bisogna fermarsi
  ^
1 31 33 38 62 69 93 97 98 99     31       2         Si puo raggiungere il prossimo,
     ^                                              si va anvanti
1 31 33 38 62 69 93 97 98 99     38       3         Bisogna fermarsi
        ^
1 31 33 38 62 69 93 97 98 99     62       4         Bisogna fermarsi
           ^
1 31 33 38 62 69 93 97 98 99     69       5         Bisogna fermarsi
              ^
1 31 33 38 62 69 93 97 98 99     69       5         Si va avanti
                 ^
1 31 33 38 62 69 93 97 98 99     69       5         Si va avanti
                 ^
1 31 33 38 62 69 93 97 98 99     69       5         Si va avanti
                    ^
1 31 33 38 62 69 93 97 98 99     69       5         Si va avanti
                       ^
1 31 33 38 62 69 93 97 98 99     99       6         Bisogna fermarsi perchè altrimenti
                          ^                         non si riesce ad arrivare a 100
\end{verbatim}
Sono state quindi necessarie \textbf{sei} fermate.
Questo algoritmo ha complessità $O(n)$ ed è quello ottimo.

\Codice

\includegraphics{code.pdf}