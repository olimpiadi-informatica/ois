% autore: Giulio Carlassare

\createsection{\Codice}{{\small{$\blacksquare$}} \normalsize Codice C++}
\createsection{\SolNQ}{{\small{$\blacksquare$}} \normalsize Soluzione $O(n^2)$}
\createsection{\SolN}{{\small{$\blacksquare$}} \normalsize Soluzione $O(n)$}
\newcommand{\tab}{\hspace{15px}}

Per risolvere il problema dobbiamo controllare che ad ogni parentesi aperta ne corrisponda una chiusa dello stesso tipo.\newline
Ad esempio l'espressione \texttt{(<>)} è corretta mentre \texttt{[(])} non lo è.
\SolNQ
Una soluzione semplice e abbastanza immediata consiste nell'eliminare mano mano le coppie di parentesi aperta-chiusa adiacenti.\newline
\textbf{Esempio}\newline

\texttt{
`<({}[])>'\tab$\Rightarrow$\tab`<([])>'\tab$\Rightarrow$\tab`<()>'\tab$\Rightarrow$\tab`<>'\tab$\Rightarrow$\tab `\ '}


Questo algoritmo ha complessità $O(n^2)$ ed è sufficiente a risolvere il problema.\newline
Tuttavia è possibile risolverlo in modo più efficiente.

\SolN
Si riduca il problema considerando tutte le parentesi come se fossero dello stesso tipo (ad esempio tonde). Ora si può utilizzare una variabile \texttt{conta}: inizialmente vale $0$, poi scorrendo l'espressione verrà incrementata per ogni parentesi aperta e decrementata per ogni parentesi chiusa.\newline
Se al termine \texttt{conta==0} l'espressione è corretta. È necessario però controllare anche che durante la scansione dell'espressione \texttt{conta} non scenda mai sotto lo $0$: in questo caso ci sarebbero più parentesi chiuse che aperte.\newline
\newline
Questo però non basta a risolvere il problema originale, che presenta quattro tipi diversi di parentesi. C'è bisogno di un ``contenitore'' per ricordare quante e quali parentesi aperte sono state già analizzate. La dimensione di questo contenitore corriponderà alla variabile \texttt{conta} utilizzata in precedenza.\newline
La struttura dati che calza a pennello in questa situazione è lo stack (pila).\newline
Ogni volta che si trova una parentesi aperta la si mette in pila, mentre per le parentesi chiuse si toglie l'ultimo elemento dello stack. Queste operazioni equivalgono a in/de-crementare \texttt{conta}. In questo caso però bisogna tenere conto del tipo delle parentesi, quindi prima di eliminarne una bisogna controllare che quella aperta e chiusa siano dello stesso tipo.\newline
Anche qui è necessario controllare che lo stack non sia vuoto prima di eliminare un elemento (\texttt{conta} non deve mai essere minore di 0).\newline
Alla fine della scansione lo stack deve essere vuoto, altrimenti l'espressione è \texttt{malformata}.\newline
\newline
\textbf{Esempio}
\begin{verbatim}
stack:              Lo stack è inizialmente vuoto
espr:   <({})

stack:  <           Si aggiunge allo stack (aperta)
espr:   <({})
        ^
stack:  <(          Si aggiunge allo stack
espr:   <({})
         ^
stack:  <({         Si aggiunge allo stack
espr:   <({})
          ^
stack:  <(          È chiusa e dello stesso tipo dell'ultima letta
espr:   <({})       quindi si toglie dallo stack
           ^
stack:  <           È chiusa e dello stesso tipo dell'ultima letta
espr:   <({})       quindi si toglie dallo stack
            ^
Sono rimasti elementi nello stack quindi l'espressione è malformata.
\end{verbatim}

Siccome ogni elemento viene analizzato una sola volta l'algoritmo ha complessità $O(n)$.

\Codice

\includegraphics{code}