% autore: Vincenzo Pandolfo

\createsection{\Nquadra}{{\small{$\blacksquare$}} \normalsize Due soluzioni $O(N^2)$} 
\createsection{\Lineare}{{\small{$\blacksquare$}} \normalsize Due soluzioni $O(N)$}
Il problema chiede di trovare il numero minimo di trampolini su cui saltare per arrivare al tappetone, ovvero una qualsiasi posizione successiva all'ultimo trampolino.

\Nquadra
Un'idea a cui si può arrivare facilmente è quella di provare ricorsivamente ogni trampolino raggiungibile dalla posizione corrente. In questo modo si proveranno tutte le possibili sequenze di trampolini e si potrà selezionare la sequenza migliore. Questa soluzione ha però una complessità esponenziale ed è in grado di risolvere solo i casi di esempio.\\
Una volta implementanta questa soluzione si noterà che ogni chiamata alla funzione ricorsiva per il trampolino $i$ restituirà la soluzione migliore al sottoproblema che ha $i$ come posizione iniziale. \\È abbastanza chiaro a questo punto che la soluzione esponenziale può essere migliorata mediante l'utilizzo della \textit{programmazione dinamica top-down} memorizzando per ogni sottoproblema la soluzione migliore trovata in modo da non doverla ricalcolare ad ogni chiamata alla funzione ricorsiva.\\
Avendo al più $N$ sottoproblemi (uno per ogni possibile posizione) e al massimo $N$ stati da visitare per ognuno di essi ne deriva che la complessità di questa soluzione è pari a $O(N^2)$, sufficiente a risolvere i primi 3 subtask.

Un'altra idea interessante è quella di modellare il problema come un \textit{DAG}, cioè un grafo aciclico diretto. I nodi saranno i trampolini ed il tappetone, mentre per ogni nodo ci saranno degli archi diretti verso i trampolini raggiungibili. Dalla figura è facile notare come eseguendo una semplice visita in ampiezza sul grafo sia possibile trovare la lunghezza della sequenza migliore. \\
Avendo al massimo $N^2$ archi e sapendo che la complessità della BFS è $O(E)$ dove E rappresenta il numero di archi, ne deriva che la complessità di questa soluzione è anch'essa pari ad $O(N^2)$.

--Inserire immagine di un grafo di questo problema colorato dall'esecuzione della BFS--

\Lineare
Per ottenere il punteggio massimo è però necessario implementare una soluzione con complessità lineare. Ci sono due approcci fondamentali per farlo: dal punto di vista dei trampolini e dal punto di vista di chi salta.

Per il primo approccio l'idea è quella di segnare, per ogni trampolino, la distanza dall'inizio in un array di dimensione $N+1$, cioè $N$ trampolini più il tappetone.\\Si inizializza la distanza del primo trampolino a $0$ e quindi, per ogni trampolino $i$, ad ogni trampolino $j$ raggiungibile da $i$ verrà assegnata una distanza pari a $distanza[i] + 1$. Siccome la prima distanza assegnata ad ogni posizione sarà sicuramente la minore, bisogna anche memorizzare la posizione dell'ultimo trampolino a cui abbiamo assegnato una distanza. In questo modo si eviterà di riassegnare la stessa distanza o addirittura peggiorarla.\\Utilizzando questa ``frontiera'', si faranno al più $N$ assegnazioni in tempo costante, per una complessità pari dunque a $O(N)$.

Il secondo approccio, invece, si basa sul fatto che è sempre possibile trovare una soluzione ottima tale che ad ogni passo il trampolino seguente sia quello che permette di arrivare il più lontano possibile al passo successivo.
Per verificare l'esistenza di una tale soluzione, si dimostrerà che è possibile trasformare una qualsiasi soluzione ottima in una soluzione che abbia la proprietà che si è definito utilizzando lo stesso numero di trampolini.

Sia $s$ una soluzione ottima, cioè una sequenza minima di trampolini.\\
Se $s$ non ha la proprietà che cerchiamo, allora ci sarà almeno un trampolino $x$ in $s$ che abbia come successivo un trampolino $y'$ per cui $y' + E[y'] < y + E[y]$ dove $y'$ e $y$ sono entrambi trampolini raggiungibili da $x$ (cioè compresi tra $x + 1$ e $x + E[x]$) e $y$ è il trampolino massimo (cioè che per ogni trampolino $i$ raggiungibile da $x$, $y + E[y] \geq i + E[i]$)\\
Siccome $y + E[y] > y' + E[y']$, il trampolino $z$ successivo a $y'$ sarà  sicuramente raggiungibile anche da $y$.\\Poiché abbiamo un numero finito di trampolini, è possibile effettuare questa correzione un numero finito di volte in modo da creare una nuova soluzione ottima $t$ che abbia la nostra proprietà.

L'implementazione è molto semplice. Avendo dimostrato l'esistenza di una soluzione che ad ogni passo scelga il trampolino che permette di andare il più lontano possibile, semplicemente partendo dalla posizione iniziale $0$ si andrà ogni volta al trampolino massimo e si conteranno i trampolini visitati. Per evitare di controllare più volte gli stessi trampolini, utilizziamo anche qui una frontiera: se un trampolino era raggiungibile anche al passo precedente, allora sarà sicuramente peggiore del trampolino massimo che cerchiamo. Dovendo fare al più $N$ controlli, si evince che la complessità di questa soluzione è $O(N)$. In codice:

\colorbox{white}{\makebox[.99\textwidth][l]{\includegraphics[scale=.8]{codice_lineare.pdf}}}