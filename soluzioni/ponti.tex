% autore: Andrea Battistello

\createsection{\Implementazione}{{\small{$\blacksquare$}} \normalsize Implementazione}
\createsection{\Codice}{{\small{$\blacksquare$}} \normalsize Codice di esempio}

Per risolvere questo problema cominciamo definendo il concetto di componente connessa. 

Una \textit{componente connessa} C di un grafo è un insieme \textbf{massimale} di nodi tali per cui esiste almeno un percorso che collega ogni coppia di nodi appartententi a C. 

Nel nostro caso, in modo abbastanza intuitivo, una componente connessa è formata da tutte le isole che si possono raggiungere partendo da un'isola qualsiasi con i ponti disponibili. 

Per definizione, quindi, se due isole appartengono alla stessa componente connessa non c'è bisogno di costruire nessun ponte, pertanto se tutte le isole appartenessero ad un'unica componente la risposta è banalmente 0.


Nel caso più generale in cui ci sono più componenti connesse, il problema diventa quello di capire come collegarle tra di loro.
Se trattiamo ogni componente connessa come un super-nodo che contiene al suo interno tutti nodi di cui è composta, abbiamo che il numero minimo di ponti che servono per collegare tra di loro questi super-nodi è esattamente uguale al numero di super-nodi (componenti connesse) - 1. 

Il problema diventa quindi quello di contare il numero di componenti connesse.

\Implementazione

L'idea per contare velocemente il numero di componenti connesse in un grafo è di iniziare dal nodo 1 e fare una visita in profondità (o in ampiezza) marchiando come visitati tutti i nodi che in qualche modo sono collegati con il nodo 1 (cioè tutti i nodi della componente connessa in cui appartiene il nodo 1). 

Dopodiché, per ogni nodo successivo al primo ci possono essere due casi:
\begin{itemize}
\item \textbf{Il nodo è già stato visitato:}

Questo vuol dire che questo nodo fa parte di una componente connessa già contata precedentemente, quindi lo ignoriamo e procediamo al prossimo nodo.

\item \textbf{Il nodo non è stato visitato:}

Abbiamo trovato una nuova componente connessa. Incrementiamo il nostro contatore ed esploriamo tutti i nodi a cui si può accedere partendo da questo nodo (marchiandoli come visitati). Procediamo poi al prossimo nodo.
\end{itemize}

Procedendo iterativamente per tutti i nodi del grafo otteniamo così il numero di componenti connesse in $O$($N$).

\Codice



\includegraphics[scale=1]{codice.pdf}