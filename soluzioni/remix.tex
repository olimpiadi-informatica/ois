% autore: Andrea Pantieri

\createsection{\Code}{{\small{$\blacksquare$}} \normalsize Codice C++}
L'idea per risolvere questo problema è controllare la stringa \textit{remix} da sinistra verso destra, utilizzando un singolo ciclo, e aggiungere alla stringa \textit{testo} solo le parti che non sono gli \textit{effetti sonori}.\newline
Questa procedura ha una complessità O(N), perciò è la più efficiente.
Per spiegare meglio, se si trova una sottostringa uguale a "PaH" o a "TuNZ" non bisogna aggiungerla, altrimenti sì.\newline
Tra due parole è sempre presente uno o più effeti sonori ad indicare che lì ci va aggiunto uno spazio, per capire quando aggiungerne uno si necessita dell'utilizzo di una variabile booleana ad indicare la fine di una parola corretta e l'inizio di un effetto sonoro.\newline
Occorre tenere salvato in una variabile l'indice corrente in qui bisogna scrivere la successiva lettera.\newline\newline
Dopo aver fatto queste considerazioni il codice risulta semplice da implementare:\newline
\begin{verbatim}
    Variabile per l'indice corente, inizializzata a 0, dove scrivere (letteraCorrente)
    Variabile per la gestione degli spazi, inizializzata a false (spazio)
    
    Ciclo da 0 a N - 1
        Se non c'è un effetto sonoro
            Aggiungo la lettera attuale a testo
            Incremento di 1 letteraCorrente
            Assegno a spazio il valore true
        Altrimenti
            Se la l'effetto sonoro è "PaH"
                Aumento l'indice del ciclo di 2
            Altrimenti
                Aumento l'indice del ciclo di 3
            
            Se spazio è true
                Aggiungo uno spazio a testo
                Incremento di 1 letteraCorrente
                Assegno a spazio il valore false
\end{verbatim}
\Code
\includegraphics{code}