% autore: Giulio Carlassare

\createsection{\SolNQ}{{\small{$\blacksquare$}} \normalsize Soluzione $O(N^2)$}
\createsection{\SolNQOtt}{{\small{$\blacksquare$}} \normalsize Soluzione $O(N^2)$ ottimizzata}
\createsection{\Codice}{{\tiny{$\blacksquare$}} \small Codice C++}
\createsection{\SolLin}{{\small{$\blacksquare$}} \normalsize Soluzione lineare}

\SolNQ
Il metodo più immediato per risolvere il problema è quello di calcolare il numero di grattacieli visibili dalla cima di ogni costruzione, e poi prendere il massimo.\newline
È possibile quindi scrivere una funzione che calcoli i grattacieli visibili sulla destra e sulla sinistra del grattacielo specificato.

Si utilizzi una variabile \texttt{limite} per memorizzare l'altezza al di sotto della quale i grattacieli sono coperti.\newline
Inizialmente il \texttt{limite} sarà uguale a $0$. Ogni qualvolta si troverà un grattacielo con le caratteristiche per coprirne altri, questa altezza verrà salvata in \texttt{limite}.

Considerando quanto sopra detto, per ogni grattacielo si contino i grattacieli visibili sulla destra (\texttt{dx}) e quelli visibili sulla sinistra (\texttt{sx}): in totale si potranno vedere $\texttt{sx} + \texttt{dx} + 1$ per contare anche il grattacielo su cui si trova in quel momento Giorgio.

Tra i valori così calcolati verrà stampato in output il \textbf{maggiore}.

Questo algoritmo ha complessità $O(N^2)$ e non è sufficiente a prendere il punteggio massimo.

\Codice


\SolNQOtt
Si può notare che il grattacielo con la vista migliore sarà \textbf{sicuramente} uno di quelli più alti (cioè uno di quelli che sono tutti alti uguali e la cui altezza è la maggiore nella fila di grattacieli).\newline
Si cerchino inizialmente i grattacieli con altezza maggiore e per ognuno di essi si ripeta il procedimento spiegato in precedenza, salvando poi il maggiore.

\SolLin
Considerando meglio la situazione, ci si accorge che \textbf{solamente un altro dei grattacieli più alti può coprire la vista} se siamo in uno dei più alti, dato che deve essere almeno alto quanto quello su cui si trova l'osservatore.\newline
Da questo si deduce quindi che, considerando $max$ come l'altezza massima che possiamo trovare nello skyline, trovandoci su un grattacielo alto $max$ saranno visibili tutti i grattacieli fino al sucessivo alto $max$.

Esempio

\includegraphics{esempio.pdf}

Questo algoritmo ha \textbf{complessità lineare} ed è quello ottimale.

\Codice