% autore: Carlo Buccisano

\createsection{\CodiceDP}{{\small{$\blacksquare$}} \normalsize Codice C++ (DP): }
\createsection{\CodiceEsp}{{\small{$\blacksquare$}} \normalsize Codice C++ (Bruteforce): }
\createsection{\Esponenziale}{{\small{$\blacksquare$}} \normalsize Soluzione Esponenziale}
\createsection{\SolDP}{{\small{$\blacksquare$}} \normalsize Soluzione Dinamica}
\newcommand{\tab}{\hspace{30px}}

Il problema chiede di calcolare il massimo numero di "X" che si può inserire in una griglia $N \times M$ in modo tale da non realizzare mai tris, ovvero 3 (o più) "X" consecutive in orizzontale, verticale o diagonale. Si analizza prima una soluzione bruteforce esponenziale, che poi si potrà ottimizzare attraverso la programmazione dinamica top-down.\newline

\Esponenziale

Per realizzare una bruteforce, che quindi prova tutti i casi possibili e tiene traccia del massimo, si può pensare che in ogni casella può essere messa o non messa la X. Ovviamente nel momento in cui viene inserita la X bisognerà controllare che non si stiano realizzando tris. La complessità è $O(2^{N \cdot M})$  dunque quando $N \times M > 20$ la soluzione sarà troppo lenta. La soluzione esponenziale risolve i primi 4 subtask, totalizzando 90 punti su 100. \newline

\CodiceEsp
\colorbox{white}{\makebox[.99\textwidth][l]{\includegraphics[scale=.8]{codiceEsp.pdf}}}

\SolDP

L'osservazione cardine per ridurre la complessità del problema, e riuscire quindi a realizzare una dinamica top down, è questa:\newline
Quando si inserisce una X in una casella generica $(i, j)$ si deve controllare che non realizzi un tris. Questa condizione, però, dipende solo dalle X che sono state inserite nella riga attuale e nelle due precedenti, e non dall'intera griglia. Dunque, anche se si è riempita la parte sovrastante della griglia in modo diverso, è sufficiente che nella riga attuale e nelle due precedenti ci siano esattamente le stesse X di prima e si avrà lo stesso risultato calcolato in precedenza.
Detto questo è facile capire che lo \texttt{stato} della Dp è \texttt{ (Numero riga, Numero colonna, Riga attuale, Riga precedente, Riga prima della precedente) }.
Per rappresentare le righe si possono utilizzare i \texttt{bitmask}. I bitmask sono interi a 32 bit dove ogni bit può essere $0$, spento, o $1$, acceso. Il bit nella posizione $i$ del bitmask sarà $1$ se in quella posizione si è inserita la X, o $0$ altrimenti. 
Considerando che nella funzione ricorsiva non si fa altro che provare a mettere o non mettere la X, quindi $O(1)$, e che vi sono $O(N \cdot M \cdot 2^{3M} )$ stati, la complessità finale è $O(N \cdot M \cdot 2^{3M} )$, che permette di risolvere completamente il problema.\newline
N.B. : per ridurre maggiormente la complessità sia come tempo sia  come spazio, se M>N si scambi N con M, dal momento che il problema è simmetrico rispetto ad essi.\newline

\CodiceDP
\colorbox{white}{\makebox[.99\textwidth][l]{\includegraphics[scale=.8]{codiceDP.pdf}}}