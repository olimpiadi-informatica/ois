% autore: Andrea Pantieri

\createsection{\SolRadLog}{{\small{$\blacksquare$}} \normalsize Una soluzione $O(\sqrt{N}*\log{N})$, efficiente}
\createsection{\SolTrucco}{{\small{$\blacksquare$}} \normalsize Una soluzione con stessa complessità, ma con una variazione}
\createsection{\CodeRadLog}{{\small{$\blacksquare$}} \normalsize Codice  $O(\sqrt{N}*\log{N})$ C++}
\createsection{\CodePoca}{{\small{$\blacksquare$}} \normalsize Codice con variazione C++}
\SolRadLog
L'idea per risolvere questo problema è di provare tutte le basi da 2 fino a $\sqrt{N}$, estremi inclusi, con ciascun esponente e tenersi salvato la massima potenza <= a N.
Le basi non possono superare $\sqrt{N}$, perchè la massima potenza che può rimanere entro il limite N è $\sqrt{N}^2$.
\newline
Questo algoritmo, visto che ripete un ciclo $\sqrt{N}$ volte ha una complessita $O(\sqrt{N}*\log{N})$.
\newline
\SolTrucco
Un altro modo per risolvere il problema è osservare che nel caso peggiore, ovvero con N = 100 000, la potenza più piccola maggiore di N è $2^{17}$, perciò basta provare solo gli esponenti da 2 a 16 e per ciascuno di essi si provano tutte le basi finchè la potenza non supera N.
\newline
\CodeRadLog
\includegraphics{codeEff}
\CodePoca
\includegraphics{codePoca}