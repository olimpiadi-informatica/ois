% \usepackage{xcolor}
% \usepackage{afterpage}
\usepackage{pifont,mdframed}
% \usepackage[bottom]{footmisc}

\makeatletter
\gdef\this@inputfilename{input.txt}
\gdef\this@outputfilename{output.txt}
\makeatother

\createsection{\Grader}{Grader di prova}

\newcommand{\inputfile}{\texttt{input.txt}}
\newcommand{\outputfile}{\texttt{output.txt}}

\newenvironment{warning}
  {\par\begin{mdframed}[linewidth=2pt,linecolor=gray]%
    \begin{list}{}{\leftmargin=1cm
                   \labelwidth=\leftmargin}\item[\Large\ding{43}]}
  {\end{list}\end{mdframed}\par}
  
Consideriamo il seguente algoritmo, che prende in ingresso un intero positivo $N$:
\begin{enumerate}[nolistsep, itemsep=2mm]
\item Se $N$ vale $1$, l’algoritmo termina.
\item Se $N$ è pari, dividi $N$ per $2$, altrimenti (se $N$ è dispari) moltiplicalo per $3$ e aggiungi $1$.
\end{enumerate}

Per esempio, applicato al valore $N=6$, l’algoritmo produce la seguente sequenza (di lunghezza $9$, contando anche il valore iniziale $N=6$ e il valore finale $1$):
$$6, 3, 10, 5, 16, 8, 4, 2, 1.$$
La congettura di Collatz, chiamata anche ``congettura $3N+1$'', afferma che l’algoritmo appena descritto termina per qualsiasi valore $N$; in altri termini, preso un qualsiasi numero intero maggiore di $1$ applicare la regola numero 2 conduce sempre al numero $1$.

Riferendosi a questa celebre congettura il famoso matematico Erdős ha commentato che questioni semplici ma elusive come questa mettono in evidenza quanto poco sia facile accedere ai misteri del ``grande Libro''.

Giorgio sta cercando di dimostrare la congettura, ed è interessato alla lunghezza della sequenza. Il vostro compito è quello di aiutare Giorgio scrivendo una funzione che, ricevuto in ingresso il numero $N$, calcoli la lunghezza della sequenza che si ottiene a partire da $N$, seguendo le regole spiegate.

\InputFile
Il file \inputfile{} è composto da una riga contenente l'intero $N$.

\OutputFile
Il file \outputfile{} è composto da una sola riga contenente la lunghezza della sequenza a partire da $N$.

\Implementation
Dovrai sottoporre esattamente un file con estensione \texttt{.c}, \texttt{.cpp} o \texttt{.pas}.

\begin{warning}
Tra gli allegati a questo task troverai un template (\texttt{collatz.c}, \texttt{collatz.cpp}, \texttt{collatz.pas}) con un esempio di implementazione da completare.
\end{warning}

Se sceglierai di utilizzare il template, dovrai implementare la seguente funzione:
\begin{center}\begin{tabularx}{\textwidth}{|c|X|}
\hline
C/C++  & \verb|int congettura(int N);|\\
\hline
Pascal & \verb|function congettura(N: longint): longint;|\\
\hline
\end{tabularx}\end{center}
La funzione riceverà come parametro il valori $N$ e dovrà ritornare la risposta al problema, che verrà stampata sul file di output.

\pagebreak
% Assunzioni
\Constraints
\begin{itemize}[nolistsep, itemsep=2mm]
\item $2 \le N \le 1\,000 $.
\item È noto che, per qualsiasi $N$ minore di $1000$, la lunghezza della sequenza è minore di $200$.
\end{itemize}

\Scoring
Il tuo programma verrà testato su diversi test case raggruppati in subtask.
Per ottenere il punteggio relativo ad un subtask, è necessario risolvere
correttamente tutti i test relativi ad esso.

\begin{itemize}[nolistsep,itemsep=2mm]
  \item \textbf{\makebox[2cm][l]{Subtask 1} [10 punti]}: Casi d'esempio.
  \item \textbf{\makebox[2cm][l]{Subtask 2} [20 punti]}: $N \le 100$.
  \item \textbf{\makebox[2cm][l]{Subtask 3} [40 punti]}: $N \le 500$.
  \item \textbf{\makebox[2cm][l]{Subtask 4} [30 punti]}: Nessuna limitazione specifica.
\end{itemize}

% Esempi
\Examples
\begin{example}
\exmp{
6
}{ %
9
} %
\end{example}
\begin{example}
\exmp{
24
}{ %
11
} %
\end{example}