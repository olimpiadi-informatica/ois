\usepackage{xcolor}
\usepackage{afterpage}
\usepackage{pifont,mdframed}
\usepackage[bottom]{footmisc}
\usepackage[colorlinks, linkcolor=black, urlcolor=blue]{hyperref}

\makeatletter
\gdef\this@inputfilename{input.txt}
\gdef\this@outputfilename{output.txt}
\makeatother

\newcommand{\inputfile}{\texttt{input.txt}}
\newcommand{\outputfile}{\texttt{output.txt}}

\newenvironment{warning}
  {\par\begin{mdframed}[linewidth=2pt,linecolor=gray]%
    \begin{list}{}{\leftmargin=1cm
                   \labelwidth=\leftmargin}\item[\Large\ding{43}]}
  {\end{list}\end{mdframed}\par}

A seguito di un violento maremoto alcuni dei ponti che collegano le $N$ isole dell'arcipelago Nowhere sono stati distrutti e il governo deve correre ai ripari per non lasciare che alcune isolette rimangano isolate e irraggiungibili.

\begin{figure}[H]
\includegraphics[width = \textwidth]{ponte_cropped.jpg}
\centering\textcolor{gray}{\small{Ponte dell'isola Kouri, in Giappone. Immagine originale: \url{http://www.panoramio.com/photo/95167664}.}}
\end{figure}

Il governo dell'arcipelago Nowhere ha quindi assunto Giorgio per determinare quale è il minimo numero di ponti che è necessario costruire in aggiunta agli $M$ rimasti affinchè l'arcipelago sia di nuovo connesso, ovvero sia possibile da ogni isola raggiungere tutte le altre isole. Aiuta Giorgio a svolgere il suo compito!

\Implementation
Dovrai sottoporre esattamente un file con estensione \texttt{.c}, \texttt{.cpp} o \texttt{.pas}.

\begin{warning}
Tra gli allegati a questo task troverai un template (\texttt{ponti.c}, \texttt{ponti.cpp}, \texttt{ponti.pas}) con un esempio di implementazione da completare.
\end{warning}

Se sceglierai di utilizzare il template, dovrai implementare la seguente funzione:
\begin{center}\begin{tabularx}{\textwidth}{|c|X|}
\hline
C/C++  & \verb|int costruisci(int N, int M, int da[], int a[]);|\\
\hline
Pascal & \verb|function costruisci(N, M: longint; var da, a: array of longint): longint;|\\
\hline
\end{tabularx}\end{center}

In cui:
\begin{itemize}[nolistsep]
  \item L'intero $N$ rappresenta il numero di isole che formano l'arcipelago.
  \item L'intero $M$ rappresenta il numero di ponti rimasti intatti dopo il maremoto.
  \item I due array \texttt{da} e \texttt{a}, indicizzati da $0$ a $M-1$, contenenti all'indice $i$ le due isole collegate dal ponte $i$.
\end{itemize}

\InputFile
Il file \inputfile{} è composto da due righe. La prima riga contiene i due interi $N$ e $M$. Le successive $M$ righe contengono due interi ciascuna, gli indici \texttt{da[$i$]}, \texttt{a[$i$]} delle isole collegate dall'$i$-esimo ponte.

\OutputFile
Il file \outputfile{} è composto da un'unica riga contenente un unico intero, la risposta a questo problema.

% Assunzioni
\Constraints
\begin{itemize}[nolistsep, itemsep=2mm]
	\item $1 \le N \le 10\,000$.
	\item $0 \le M \le 100\,000$.
	\item $0 \le \texttt{da}[i], \texttt{a}[i] < N$ per ogni $i=0\ldots M-1$.
	\item Per ogni coppia di isole esiste al più un ponte che le collega, e i ponti non vengono ripetuti nell'input.
	\item Nessun ponte collega un'isola a se stessa.
	\item Le isole sono numerate a partire da 0.
	\item Se l'arcipelago è già connesso, rispondere il valore 0.
\end{itemize}

\Scoring
Il tuo programma verrà testato su diversi test case raggruppati in subtask.
Per ottenere il punteggio relativo ad un subtask, è necessario risolvere
correttamente tutti i test relativi ad esso.

\begin{itemize}[nolistsep,itemsep=2mm]
  \item \textbf{\makebox[2cm][l]{Subtask 1} [10 punti]}: Casi d'esempio.
  \item \textbf{\makebox[2cm][l]{Subtask 2} [20 punti]}: $N \leq 10$.
  \item \textbf{\makebox[2cm][l]{Subtask 3} [40 punti]}: $N \leq 100$.
  \item \textbf{\makebox[2cm][l]{Subtask 4} [30 punti]}: Nessuna limitazione specifica.
\end{itemize}

% Esempi
\Examples
\begin{example}
\exmp{
2 0
}{%
1
}%
\end{example}
\begin{example}
\exmp{
4 2
1 3
3 2
}{%
1
}%
\end{example}


\Explanation
Nel \textbf{primo caso di esempio} è sufficiente costruire un ponte tra le isole 0 e 1.\\[2mm]
Nel \textbf{secondo caso di esempio} è sufficiente costruire un ponte tra le isole 0 e 2.