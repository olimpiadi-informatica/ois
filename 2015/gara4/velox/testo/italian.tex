\usepackage{xcolor}
\usepackage{afterpage}
\usepackage{pifont,mdframed}
\usepackage[bottom]{footmisc}

\makeatletter
\gdef\this@inputfilename{input.txt}
\gdef\this@outputfilename{output.txt}
\makeatother

\newcommand{\inputfile}{\texttt{input.txt}}
\newcommand{\outputfile}{\texttt{output.txt}}

\newenvironment{warning}
  {\par\begin{mdframed}[linewidth=2pt,linecolor=gray]%
    \begin{list}{}{\leftmargin=1cm
                   \labelwidth=\leftmargin}\item[\Large\ding{43}]}
  {\end{list}\end{mdframed}\par}

	Gabriele vuole mettere alla prova la sua nuova utilitaria sportiva, e sta quindi cercando il tragitto più lungo possibile da fare in un'unica graduale accelerata. Il compito è reso arduo dalla presenza di un complicato sistema di limiti di velocità e sensi unici, che Gabriele non può assolutamente permettersi di violare.

	La zona di Brescia è composta di $N$ incroci tra cui corrono $M$ tratti di strada di pari lunghezza, ciascuno con il suo limite di velocità $V_i$. Trova il tragitto più lungo composto da tratti di strada con limiti di velocità strettamente crescenti!

\Implementation
Dovrai sottoporre esattamente un file con estensione \texttt{.c}, \texttt{.cpp} o \texttt{.pas}.

\begin{warning}
Tra gli allegati a questo task troverai un template (\texttt{velox.c}, \texttt{velox.cpp}, \texttt{velox.pas}) con un esempio di implementazione da completare.
\end{warning}

Se sceglierai di utilizzare il template, dovrai implementare la seguente funzione:
\begin{center}\begin{tabularx}{\textwidth}{|c|X|}
\hline
C/C++  & \verb|int accelera(int N, int M, int da[], int a[], V[]);|\\
\hline
Pascal & \verb|function accelera(N, M: longint; var da, a, V: array of longint): longint;|\\
\hline
\end{tabularx}\end{center}
In cui:
\begin{itemize}[nolistsep]
  \item L'intero $N$ rappresenta il numero di incroci, indicizzati da $0$ a $N-1$.
  \item L'intero $M$ rappresenta il numero di tratti di strada, indicizzati da $0$ a $M-1$.
  \item Gli array \texttt{da}, \texttt{a}, \texttt{V}, indicizzati da $0$ a $M-1$, contengono la descrizione dei tratti di strada. Più precisamente, per ogni $i$ vi è un tratto di strada a senso unico dall'incrocio \texttt{da[$i$]} all'incrocio \texttt{a[$i$]} con limite di velocità \texttt{V[$i$]}.
  \item La funzione dovrà restituire la massima lunghezza per un percorso lungo il quale i limiti di velocità sono strettamente crescenti, che verrà stampata sul file di output.
\end{itemize}

\InputFile
Il file \inputfile{} è composto da $M+1$ righe. La prima riga contiene i due interi $M$, $N$. Le successive $M$ righe contengono ciascuna i tre interi \texttt{da[$i$]}, \texttt{a[$i$]}, \texttt{V[$i$]} separati da uno spazio.

\OutputFile
Il file \outputfile{} è composto da un'unica riga contenente un unico intero, la risposta a questo problema.

\pagebreak
% Assunzioni
\Constraints
\begin{itemize}[nolistsep, itemsep=2mm]
	\item $2 \le N \le 10\,000$.
	\item $1 \le M \le 100\,000$.
	\item $1 \le V_i \le 1\,000\,000$ per ogni $i=0\ldots N-1$.
	\item I tratti di strada sono tutti validi e non ripetuti (non hanno lo stesso \texttt{da} e lo stesso \texttt{a}).
	\item Le strade sono da considerarsi a senso unico, a meno che vengano indicate nell'input ``due volte'' (una per senso di marcia, con \texttt{da} e \texttt{a} invertiti). I limiti di velocità nei due sensi possono differire.
\end{itemize}

\Scoring
Il tuo programma verrà testato su diversi test case raggruppati in subtask.
Per ottenere il punteggio relativo ad un subtask, è necessario risolvere
correttamente tutti i test relativi ad esso.

\begin{itemize}[nolistsep,itemsep=2mm]
  \item \textbf{\makebox[2cm][l]{Subtask 1} [10 punti]}: Casi d'esempio.
  \item \textbf{\makebox[2cm][l]{Subtask 2} [40 punti]}: $N \leq 10$.
  \item \textbf{\makebox[2cm][l]{Subtask 3} [30 punti]}: $N \leq 100$.
  \item \textbf{\makebox[2cm][l]{Subtask 4} [20 punti]}: Nessuna limitazione specifica.
\end{itemize}

% Esempi
\Examples
\begin{example}
\exmp{
3 5
0 1 90
0 2 90
1 0 90
1 2 90
2 0 90
}{%
1
}%
\end{example}
\begin{example}
\exmp{
5 6
0 1 60
1 2 110
2 1 90
2 3 130
3 0 70
3 4 50
}{%
3
}%
\end{example}


\Explanation
Nel \textbf{primo caso di esempio}, è possibile fare un solo tratto.\\[2mm]
Nel \textbf{secondo caso di esempio} è possibile fare tre tratti in vari modi, 0--1--2--3 oppure 2--1--2--3, ma non è possibile farne quattro.

\begin{figure}
	\begin{center}
		\begin{minipage}[t]{0.3\textwidth}
			\includegraphics[width=\textwidth]{asy_limiti/esempio2_1.pdf}
		\end{minipage}
		\quad\quad\quad
		\begin{minipage}[t]{0.3\textwidth}
			\includegraphics[width=\textwidth]{asy_limiti/esempio2_2.pdf}
		\end{minipage}
	\end{center}
	\caption{Percorsi ottimali per il secondo caso di esempio.}
\end{figure}
