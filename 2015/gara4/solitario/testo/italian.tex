% \usepackage{xcolor}
% \usepackage{afterpage}
\usepackage{pifont,mdframed}
% \usepackage[bottom]{footmisc}

\makeatletter
\gdef\this@inputfilename{input.txt}
\gdef\this@outputfilename{output.txt}
\makeatother

\newcommand{\inputfile}{\texttt{input.txt}}
\newcommand{\outputfile}{\texttt{output.txt}}

\newenvironment{warning}
  {\par\begin{mdframed}[linewidth=2pt,linecolor=gray]%
    \begin{list}{}{\leftmargin=1cm
                   \labelwidth=\leftmargin}\item[\Large\ding{43}]}
  {\end{list}\end{mdframed}\par}

	Gabriele, ancora ossessionato dal gioco del tris, ha deciso che non vuole più ricorrere a un programma per vincere contro di Giorgio e ha quindi iniziato un duro programma di allenamento in solitaria. Per allenarsi, Gabriele prende una griglia da tris composta da $N \times M$ caselle, e va avanti a segnare \texttt{X} su alcune caselle finché riesce a non creare tris.

	Questo programma di allenamento, tuttavia, lo sta lasciando insoddisfatto in quanto non riesce a capire se sta andando bene oppure no. Aiuta Gabriele a valutarsi calcolando quante \texttt{X} può segnare al massimo in una griglia $N \times M$ senza creare tris!

\Implementation
Dovrai sottoporre esattamente un file con estensione \texttt{.c}, \texttt{.cpp} o \texttt{.pas}.

\begin{warning}
Tra gli allegati a questo task troverai un template (\texttt{solitario.c}, \texttt{solitario.cpp}, \texttt{solitario.pas}) con un esempio di implementazione da completare.
\end{warning}

Se sceglierai di utilizzare il template, dovrai implementare la seguente funzione:
\begin{center}\begin{tabularx}{\textwidth}{|c|X|}
\hline
C/C++  & \verb|int gioca(int N, int M);|\\
\hline
Pascal & \verb|function gioca(N, M: longint): longint;|\\
\hline
\end{tabularx}\end{center}
In cui:
\begin{itemize}[nolistsep]
  \item Gli interi $N$, $M$ rappresentano le dimensioni della griglia.
  \item La funzione dovrà restituire il numero massimo di \texttt{X} che si possono segnare senza creare tris, che verrà stampato sul file di output.
\end{itemize}

\InputFile
Il file \inputfile{} è composto da un'unica riga contenente i due interi $N$ ed $M$.

\OutputFile
Il file \outputfile{} è composto da un'unica riga contenente un unico intero, la risposta a questo problema.

% Assunzioni
\Constraints
\begin{itemize}[nolistsep, itemsep=2mm]
	\item $1 \le N, M \le 10$.
	\item $N \times M \le 36$.
\end{itemize}

\pagebreak
\Scoring
Il tuo programma verrà testato su diversi test case raggruppati in subtask.
Per ottenere il punteggio relativo ad un subtask, è necessario risolvere
correttamente tutti i test relativi ad esso.

\begin{itemize}[nolistsep,itemsep=2mm]
  \item \textbf{\makebox[2cm][l]{Subtask 1} [10 punti]}: Casi d'esempio.
  \item \textbf{\makebox[2cm][l]{Subtask 2} [20 punti]}: $N \leq 2$.
  \item \textbf{\makebox[2cm][l]{Subtask 3} [30 punti]}: $N \times M \le 16$.
  \item \textbf{\makebox[2cm][l]{Subtask 4} [30 punti]}: $N \times M \le 25$.
  \item \textbf{\makebox[2cm][l]{Subtask 5} [10 punti]}: Nessuna limitazione specifica.
\end{itemize}

% Esempi
\Examples
\begin{example}
\exmp{
2 6
}{%
8
}%
\end{example}
\begin{example}
\exmp{
3 3
}{%
6
}%
\end{example}

\Explanation
Nel \textbf{primo caso di esempio}, si può riempire tutta la griglia tranne la seconda e la penultima colonna.\\[2mm]
Nel \textbf{secondo caso di esempio}, la configurazione massima è:
\begin{figure}[h]
	\centering\includegraphics[scale = 1.5]{asy_solitario/esempio.pdf}
\end{figure}
