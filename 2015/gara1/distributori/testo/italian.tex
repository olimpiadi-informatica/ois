\usepackage{xcolor}
\usepackage{afterpage}
\usepackage{pifont,mdframed}
\usepackage[bottom]{footmisc}

\makeatletter
\gdef\this@inputfilename{input.txt}
\gdef\this@outputfilename{output.txt}
\makeatother

\newcommand{\inputfile}{\texttt{input.txt}}
\newcommand{\outputfile}{\texttt{output.txt}}

\newenvironment{warning}
  {\par\begin{mdframed}[linewidth=2pt,linecolor=gray]%
    \begin{list}{}{\leftmargin=1cm
                   \labelwidth=\leftmargin}\item[\Large\ding{43}]}
  {\end{list}\end{mdframed}\par}

	Gabriele ha appena preso la patente, e decide di invitare tutti i suoi amici a fare una gita. Dato che il viaggio è lungo ben $K$ chilometri, sa che forse dovrà fermarsi a fare il pieno di benzina: a tal proposito Gabriele ha segnato a che distanza dalla partenza ci sono gli $N$ distributori che si trovano lungo il tragitto. Sapendo che la sua macchina fa al massimo $M$ chilometri con un pieno, e che alla partenza ha già il serbatoio pieno, aiuta Gabriele a pianificare i rifornimenti di modo da fare benzina il minor numero possibile di volte.

\InputFile
Il file \inputfile{} è composto da due righe. La prima riga contiene i tre interi $N$, $M$, $K$ separati da uno spazio. La seconda riga contiene $N$ interi separati da uno spazio, le distanze $D_i$ dei distributori in ordine crescente.

\OutputFile
Il file \outputfile{} è composto da un'unica riga contenente un unico intero, la risposta a questo problema.

\Implementation
Dovrai sottoporre esattamente un file con estensione \texttt{.c}, \texttt{.cpp} o \texttt{.pas}.

\begin{warning}
Tra gli allegati a questo task troverai un template (\texttt{distributori.c}, \texttt{distributori.cpp}, \texttt{distributori.pas}) con un esempio di implementazione da completare.
\end{warning}

Se sceglierai di utilizzare il template, dovrai implementare la seguente funzione:
\begin{center}\begin{tabularx}{\textwidth}{|c|X|}
\hline
C/C++  & \verb|int rifornisci(int N, int M, int K, int D[]);|\\
\hline
Pascal & \verb|function rifornisci(N, M, K: longint; var D: array of longint): longint;|\\
\hline
\end{tabularx}\end{center}
In cui:
\begin{itemize}[nolistsep]
  \item L'intero $N$ rappresenta il numero di distributori.
  \item L'intero $M$ rappresenta il massimo numero di chilometri con un pieno che la macchina è in grado di fare.
  \item L'intero $K$ rappresenta la lunghezza totale del viaggio in chilometri.
  \item L'array \texttt{D}, indicizzato da $0$ a $N-1$, contiene le distanze dei distributori dalla partenza in ordine crescente.
  \item La funzione dovrà restituire il minor numero di rifornimenti che è necessario fare, che verrà stampato sul file di output.
\end{itemize}

\pagebreak
% Assunzioni
\Constraints
\begin{itemize}[nolistsep, itemsep=2mm]
\item $1 \le N \le 100\,000$.
\item $1 \le M \le K \le 1\,000\,000$.
\item $1 \le D_i < D_{i+1} < K$ per ogni $i=0\ldots N-2$.
\item È sempre possibile raggiungere la destinazione: la distanza tra due distributori successivi non supera mai $M$, il numero di chilometri che la macchina è in grado di percorrere con un pieno.
\end{itemize}

\Scoring
Il tuo programma verrà testato su diversi test case raggruppati in subtask.
Per ottenere il punteggio relativo ad un subtask, è necessario risolvere
correttamente tutti i test relativi ad esso.

\begin{itemize}[nolistsep,itemsep=2mm]
  \item \textbf{\makebox[2cm][l]{Subtask 1} [10 punti]}: Casi d'esempio.
  \item \textbf{\makebox[2cm][l]{Subtask 2} [20 punti]}: $N \leq 10$.
  \item \textbf{\makebox[2cm][l]{Subtask 3} [40 punti]}: $N, K \leq 1000$.
  \item \textbf{\makebox[2cm][l]{Subtask 4} [30 punti]}: Nessuna limitazione specifica.
\end{itemize}

% Esempi
\Examples
\begin{example}
\exmp{
5 50 100
29 35 50 77 83
}{%
1
}%
\end{example}
\begin{example}
\exmp{
10 30 100
1 31 33 38 62 69 93 97 98 99
}{%
6
}%
\end{example}


\Explanation
Nel \textbf{primo caso di esempio}, è sufficiente fermarsi al distributore al chilometro 50.\\[2mm]
Nel \textbf{secondo caso di esempio}, è necessario fermarsi ai distributori ai chilometri 1, 31, 33 o 38, 62, 69, e infine uno tra quelli ai chilometri 93, 97, 98, 99.
