\usepackage{xcolor}
\usepackage{afterpage}
\usepackage{pifont,mdframed}
\usepackage[bottom]{footmisc}

\makeatletter
\gdef\this@inputfilename{input.txt}
\gdef\this@outputfilename{output.txt}
\makeatother

\newcommand{\inputfile}{\texttt{input.txt}}
\newcommand{\outputfile}{\texttt{output.txt}}

\newenvironment{warning}
  {\par\begin{mdframed}[linewidth=2pt,linecolor=gray]%
    \begin{list}{}{\leftmargin=1cm
                   \labelwidth=\leftmargin}\item[\Large\ding{43}]}
  {\end{list}\end{mdframed}\par}

	Giorgio ha scritto un programma contenente una serie di espressioni molto elaborate, formate ciascuna da un gran numero di parentesi di tutti i tipi, e cioè:
	\begin{itemize}
		\item angolate: \texttt{'<'} e \texttt{'>'}
		\item tonde: \texttt{'('} e \texttt{')'}
		\item quadrate: \texttt{'['} e \texttt{']'}
		\item graffe: \texttt{'\{'} e \texttt{'\}'}
	\end{itemize}
	Purtroppo, quando ha provato ad eseguirlo, il compilatore gli ha detto che c'è un errore senza nemmeno dirgli in quale espressione si trova! Aiutalo controllando quali espressioni sono ben formate e quali no.

\Implementation
Dovrai sottoporre esattamente un file con estensione \texttt{.c}, \texttt{.cpp} o \texttt{.pas}.

\begin{warning}
Tra gli allegati a questo task troverai un template (\texttt{parentesi.c}, \texttt{parentesi.cpp}, \texttt{parentesi.pas}) con un esempio di implementazione da completare.
\end{warning}

Se sceglierai di utilizzare il template, dovrai implementare la seguente funzione:
\begin{center}\begin{tabularx}{\textwidth}{|c|X|}
\hline
C/C++  & \verb|int controlla(int N, char E[]);|\\
\hline
Pascal & \verb|function controlla(N: longint; var E: array of char): longint;|\\
\hline
\end{tabularx}\end{center}
In cui:
\begin{itemize}[nolistsep]
  \item L'intero $N$ rappresenta la lunghezza dell'espressione da controllare.
  \item L'array \texttt{E}, indicizzato da $0$ a $N-1$, contiene i caratteri di cui l'espressione è composta.
  \item La funzione dovrà restituire 0 se l'espressione è corretta, -1 altrimenti. Nel primo caso, sul file di output verrà stampata la stringa \texttt{'corretta'}, nel secondo caso la stringa \texttt{'malformata'}.
\end{itemize}

\InputFile
Il file \inputfile{} è composto da due righe. La prima riga contiene l'unico intero $N$. La seconda riga contiene la stringa $E$.

\OutputFile
Il file \outputfile{} è composto da un'unica riga contenente un unica parola, la risposta a questo problema.

\pagebreak
% Assunzioni
\Constraints
\begin{itemize}[nolistsep, itemsep=2mm]
	\item $1 \le N \le 10\,000$.
	\item $E_i$ è uno tra i caratteri \texttt{'\{[(<>)]\}'}.
\end{itemize}

\Scoring
Il tuo programma verrà testato su diversi test case raggruppati in subtask.
Per ottenere il punteggio relativo ad un subtask, è necessario risolvere
correttamente tutti i test relativi ad esso.

\begin{itemize}[nolistsep,itemsep=2mm]
  \item \textbf{\makebox[2cm][l]{Subtask 1} [10 punti]}: Casi d'esempio.
  \item \textbf{\makebox[2cm][l]{Subtask 2} [20 punti]}: Tutte le parentesi sono tonde.
  \item \textbf{\makebox[2cm][l]{Subtask 3} [20 punti]}: Le espressioni corrette sono formate da parentesi ordinate gerarchicamente (vedi quarto caso di esempio).
  \item \textbf{\makebox[2cm][l]{Subtask 4} [30 punti]}: $N \leq 30$.
  \item \textbf{\makebox[2cm][l]{Subtask 5} [20 punti]}: Nessuna limitazione specifica.
\end{itemize}

% Esempi
\Examples
\begin{example}
\exmp{
4
([)]
}{%
malformata
}%
\end{example}
\begin{example}
\exmp{
5
<(\{\})
}{%
malformata
}%
\end{example}
\begin{example}
\exmp{
12
()([]\{(<>)\})
}{%
corretta
}%
\end{example}
\begin{example}
\exmp{
20
\{(<><>)\{\{()[<>]<>\}\}\}
}{%
corretta
}%
\end{example}


\Explanation
Nel \textbf{primo caso di esempio}, la parentesi tonda più esterna viene chiusa prima della parentesi quadra (che è più interna), e quindi l'espressione è malformata.\\[2mm]
Nel \textbf{secondo caso di esempio}, la parentesi angolata all'inizio non viene mai chiusa.\\[2mm]
Nel \textbf{terzo e quarto caso di esempio}, tutte le parentesi sono chiuse correttamente. Inoltre nell'ultimo viene rispettata la gerarchia tra le parentesi: le angolate sono tutte più all'interno, a seguire tonde, quadre e infine graffe più all'esterno. In altre parole, una parentesi graffa non può essere aperta all'interno di un'altro tipo di parentesi; una parentesi quadra non può essere aperta all'interno di una parentesi tonda o angolata, e una parentesi tonda non può essere aperta all'interno di una parentesi angolata.
