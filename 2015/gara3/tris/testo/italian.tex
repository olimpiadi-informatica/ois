% \usepackage{xcolor}
% \usepackage{afterpage}
% \usepackage{pifont,mdframed}
% \usepackage[bottom]{footmisc}

\makeatletter
\gdef\this@inputfilename{input.txt}
\gdef\this@outputfilename{output.txt}
\makeatother

\newcommand{\inputfile}{\texttt{input.txt}}
\newcommand{\outputfile}{\texttt{output.txt}}

\newenvironment{warning}
  {\par\begin{mdframed}[linewidth=2pt,linecolor=gray]%
    \begin{list}{}{\leftmargin=1cm
                   \labelwidth=\leftmargin}\item[\Large\ding{43}]}
  {\end{list}\end{mdframed}\par}

Quasi tutti conoscono il gioco del tris, ma per completezza riportiamo qui sotto le regole:
\begin{itemize}[nolistsep, itemsep=2mm]
	\item Il gioco si svolge su una griglia $3\times 3$, inizialmente vuota.
	\item A turno, i giocatori scelgono una cella vuota e vi disegnano il proprio simbolo (una \texttt{X} o una \texttt{O}).
	\item Vince il giocatore che riesce a disporre tre dei propri simboli in linea retta orizzontale, verticale o diagonale. Se la griglia viene riempita senza che nessuno dei giocatori sia riuscito a completare una linea retta di tre simboli, il gioco finisce in parità e nessuno dei giocatori vince.
\end{itemize}

Gabriele è stanco di perdere in continuazione contro Giorgio, abilissimo giocatore di tris, perciò decide di scrivere un programma in grado di predire, conoscendo lo stato corrente della griglia, se Gabriele riuscirà a vincere indipendentemente dalle mosse di Giorgio. Gabriele gioca sempre per primo col simbolo \texttt{X}, mentre giorgio gioca sempre per secondo con il simbolo \texttt{O}.

A titolo di esempio, consideriamo la griglia
\begin{center}
	\includegraphics[width=3cm]{ttt1.pdf}
\end{center}
Contando il numero di simboli, si deduce che è il turno di Giorgio. Se Giorgio non "blocca il tris" sulla seconda riga, Gabriele avrà vittoria facile; supponiamo perciò che Giorgio blocchi il tris:
\begin{center}
	\includegraphics[width=3cm]{ttt2.pdf}
\end{center}
A questo punto Gabriele posiziona il suo simbolo nell'angolo in alto a destra.
\begin{center}
	\includegraphics[width=3cm]{ttt3.pdf}
\end{center}
Giorgio si trova ora spiazzato, in quanto dovrebbe coprire due potenziali tris, e perde la partita. Nell'esempio appena mostrato, è evidente che Giorgio perde a prescindere dalle sue mosse: se non avesse coperto il tris sulla seconda riga avrebbe perso, ma anche coprendolo viene sconfitto in poche mosse.

\Implementation
Dovrai sottoporre esattamente un file con estensione \texttt{.c}, \texttt{.cpp} o \texttt{.pas}.

\begin{warning}
Tra gli allegati a questo task troverai un template (\texttt{tris.c}, \texttt{tris.cpp}, \texttt{tris.pas}) con un esempio di implementazione da completare.
\end{warning}

Se sceglierai di utilizzare il template, dovrai implementare la seguente funzione:
\begin{center}\begin{tabularx}{\textwidth}{|c|X|}
\hline
C/C++  & \verb|int vincente(char griglia[3][3]);|\\
\hline
Pascal & \verb|function vincente(var griglia: array[0..2, 0..2] of char): longint;|\\
\hline
\end{tabularx}\end{center}
In cui:
\begin{itemize}[nolistsep]
  \item L'array \texttt{griglia} è una matrice $3\times 3$, in cui le righe e le colonne sono indicizzate a partire da 0, che rappresenta lo stato della griglia. Ogni cella della matrice contiene un punto se la corrispondente cella della griglia è vuota, una \texttt{X} se è occupata da un simbolo di Gabriele e una \texttt{O} se è occupata da un simbolo di Giorgio. 
  \item La funzione dovrà restituire 1 se Gabriele può vincere indipendemente dalle mosse di Giorgio, e 0 altrimenti. Il valore ritornato verrà stampato sul file di output.
\end{itemize}

\InputFile
Il file \inputfile{} è composto da tre righe, ognuna formata da 3 caratteri, rappresentanti la situazione della griglia.

\OutputFile
Il file \outputfile{} è composto da un'unica riga contenente il valore 1 se Gabriele può vincere indipendemente dalle mosse di Giorgio, e 0 altrimenti.

% Assunzioni
\Constraints
\begin{itemize}[nolistsep, itemsep=2mm]
	\item Esiste almeno una cella vuota.
	\item La griglia rappresenta una partita ancora in corso: non sono presenti tris.
	\item Il pareggio non è considerata una vittoria di Gabriele.
\end{itemize}

\Scoring
Il tuo programma verrà testato su diversi test case raggruppati in subtask.
Per ottenere il punteggio relativo ad un subtask, è necessario risolvere
correttamente tutti i test relativi ad esso.

\begin{itemize}[nolistsep,itemsep=2mm]
  \item \textbf{\makebox[2cm][l]{Subtask 1} [10 punti]}: Casi d'esempio.
  \item \textbf{\makebox[2cm][l]{Subtask 2} [30 punti]}: Esiste esattamente una cella vuota nella griglia.
  \item \textbf{\makebox[2cm][l]{Subtask 3} [30 punti]}: Il numero di celle vuote è $\le 4$.
  \item \textbf{\makebox[2cm][l]{Subtask 4} [30 punti]}: nessuna limitazione specifica.
\end{itemize}

% Esempi
\Examples
\begin{example}
\exmp{
.O.
.XX
...
}{%
1
}%
\end{example}
\begin{example}
\exmp{
..O
.XX
...
}{%
0
} %
\end{example}
\begin{example}
\exmp{
..O
.XX
XOO
}{%
1
}%
\end{example}


\Explanation
Il \textbf{primo caso di esempio} corrisponde all'esempio del testo.\\[2mm]
Nel \textbf{secondo caso di esempio} se Giorgio gioca bene può pareggiare.\\[2mm]
Nel \textbf{terzo caso di esempio} è sufficiente che Gabriele completi il tris sulla seconda riga.
