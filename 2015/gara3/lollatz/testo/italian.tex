% \usepackage{xcolor}
% \usepackage{afterpage}
% \usepackage{pifont,mdframed}
% \usepackage[bottom]{footmisc}

\makeatletter
\gdef\this@inputfilename{input.txt}
\gdef\this@outputfilename{output.txt}
\makeatother

\newcommand{\inputfile}{\texttt{input.txt}}
\newcommand{\outputfile}{\texttt{output.txt}}

\newenvironment{warning}
  {\par\begin{mdframed}[linewidth=2pt,linecolor=gray]%
    \begin{list}{}{\leftmargin=1cm
                   \labelwidth=\leftmargin}\item[\Large\ding{43}]}
  {\end{list}\end{mdframed}\par}

	Da qualche tempo, Giorgio si è interessato alla congettura di \emph{Lollatz}. Questa congettura afferma che, dato un quadrato perfetto $N$, ripetendo i seguenti due passaggi:
	\begin{itemize}
		\item moltiplichi il numero per la sua $(\text{cifra delle unità} - 1)$
		\item dividi per due, arrotondando per difetto
	\end{itemize}
	prima o poi si arriva a un multiplo di 10. Giorgio, tuttavia, non riesce ad interpretare i passaggi della dimostrazione del dottor Lollatz, che sembra essere composta unicamente da abbreviazioni poco leggibili, e quindi si fida poco di questo risultato. Aiutalo a dirimere i suoi dubbi scrivendo un programma che verifichi questa congettura!

\Implementation
Dovrai sottoporre esattamente un file con estensione \texttt{.c}, \texttt{.cpp} o \texttt{.pas}.

\begin{warning}
Tra gli allegati a questo task troverai un template (\texttt{lollatz.c}, \texttt{lollatz.cpp}, \texttt{lollatz.pas}) con un esempio di implementazione da completare.
\end{warning}

Se sceglierai di utilizzare il template, dovrai implementare la seguente funzione:
\begin{center}\begin{tabularx}{\textwidth}{|c|X|}
\hline
C/C++  & \verb|int afaikdiyrotfl(int N);|\\
\hline
Pascal & \verb|function afaikdiyrotfl(N: longint): longint;|\\
\hline
\end{tabularx}\end{center}
In cui:
\begin{itemize}[nolistsep]
  \item L'intero $N$ rappresenta il quadrato perfetto di partenza.
  \item La funzione dovrà restituire $-1$ se la congettura non è vera per $N$, altrimenti dovrà restituire il multiplo di 10 che si ottiene alla fine del procedimento. Il valore restituito verrà stampato sul file di output.
\end{itemize}

\InputFile
Il file \inputfile{} è composto da un'unica riga contenente l'unico intero $N$.

\OutputFile
Il file \outputfile{} è composto da un'unica riga contenente un unico intero, la risposta a questo problema.

% Assunzioni
\Constraints
\begin{itemize}[nolistsep, itemsep=2mm]
	\item $1 \le N \le 1\,000\,000$.
\end{itemize}

\Scoring
Il tuo programma verrà testato su diversi test case raggruppati in subtask.
Per ottenere il punteggio relativo ad un subtask, è necessario risolvere
correttamente tutti i test relativi ad esso.

\begin{itemize}[nolistsep,itemsep=2mm]
  \item \textbf{\makebox[2cm][l]{Subtask 1} [10 punti]}: Casi d'esempio.
  \item \textbf{\makebox[2cm][l]{Subtask 2} [20 punti]}: $N \leq 100$.
  \item \textbf{\makebox[2cm][l]{Subtask 3} [40 punti]}: $N \leq 1000$.
  \item \textbf{\makebox[2cm][l]{Subtask 4} [30 punti]}: Nessuna limitazione specifica.
\end{itemize}

% Esempi
\Examples
\begin{example}
\exmp{
4
}{%
30
}%
\end{example}
\begin{example}
\exmp{
100
}{%
100
}%
\end{example}


\Explanation
Nel \textbf{primo caso di esempio}, dopo il primo passaggio otteniamo $4 \times (4-1)/2 = 6$. Proseguiamo quindi ottenendo $6 \times (6-1)/2 = 15$. Procediamo ancora ottenendo $15 \times (5-1)/2 = 30$, e qui ci possiamo fermare. \\[2mm]
Nel \textbf{secondo caso di esempio}, il numero è già un multiplo di 10 quindi l'algoritmo si ferma subito.
