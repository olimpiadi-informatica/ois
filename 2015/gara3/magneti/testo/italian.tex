\usepackage{xcolor}
\usepackage{afterpage}
\usepackage{pifont,mdframed}
\usepackage[bottom]{footmisc}

\makeatletter
\gdef\this@inputfilename{input.txt}
\gdef\this@outputfilename{output.txt}
\makeatother

\newcommand{\inputfile}{\texttt{input.txt}}
\newcommand{\outputfile}{\texttt{output.txt}}

\newenvironment{warning}
  {\par\begin{mdframed}[linewidth=2pt,linecolor=gray]%
    \begin{list}{}{\leftmargin=1cm
                   \labelwidth=\leftmargin}\item[\Large\ding{43}]}
  {\end{list}\end{mdframed}\par}

\newcommand{\mpm}{(\textcolor{blue}{+}\textcolor{red}{-})}
\newcommand{\mmp}{(\textcolor{red}{-}\textcolor{blue}{+})}
Gabriele, nonostante la sua età, ama moltissimo giocare con i magneti. Una delle sue costruzioni preferite è il cosiddetto ``serpente di magneti'', cioè una lunga linea di magneti. Ovviamente, affinchè i magneti rimangano accostati, è necessario che al polo positivo di un magnete corrisponda sempre il polo negativo del successivo, e viceversa. Ad esempio, sono serpenti di magneti validi le successioni di magneti
$$\mpm\mpm\mpm\mpm\mpm\quad\text{e}\quad\mmp\mmp\mmp,$$
mentre non lo sono
$$\mpm\mmp\mmp\mmp\quad\text{e}\quad\mmp\mpm\mmp.$$
Data una generica successione (non necessariamente stabile) di magneti, Gabriele si chiede quale è il minimo numero di magneti da girare affinchè la successione risultante sia un serpente di magneti.

\Implementation
Dovrai sottoporre esattamente un file con estensione \texttt{.c}, \texttt{.cpp} o \texttt{.pas}.

\begin{warning}
Tra gli allegati a questo task troverai un template (\texttt{magneti.c}, \texttt{magneti.cpp}, \texttt{magneti.pas}) con un esempio di implementazione da completare.
\end{warning}

Se sceglierai di utilizzare il template, dovrai implementare la seguente funzione:
\begin{center}\begin{tabularx}{\textwidth}{|c|X|}
\hline
C/C++  & \verb|int gira(int N, char descrizione[]);|\\
\hline
Pascal & \verb|function gira(N: longint; var descrizione: array of char): longint;|\\
\hline
\end{tabularx}\end{center}
In cui:
\begin{itemize}[nolistsep]
  \item L'intero $N$ rappresenta il numero di caratteri della stringa \texttt{descrizione}.
  \item La stringa \texttt{descrizione}, array di caratteri indicizzato da $0$ a $N-1$, contiene la descrizione della situazione iniziale dei magneti, nello stesso formato degli esempi appena fatti (si veda la sezione \textbf{Esempi di input/output} per due ulteriori esemplificazioni del formato della stringa).
  \item La funzione dovrà restituire il minimo numero di magneti che è necessario girare, che verrà stampato sul file di output.
\end{itemize}

\InputFile
Il file \inputfile{} è composto da due righe. La prima riga contiene l'unico intero $N$. La seconda riga contiene la stringa \texttt{descrizione}, composta da $N$ caratteri.

\OutputFile
Il file \outputfile{} è composto da un'unica riga contenente un unico intero, il minimo numero di magneti che è necessario ruotare per rendere la successione un serpente di magneti.

\newpage
% Assunzioni
\Constraints
\begin{itemize}[nolistsep, itemsep=2mm]
	\item $4 \le N \le 100\,000$.
	\item La stringa \texttt{descrizione} è formata solo dai caratteri \texttt{(},\texttt{+},\texttt{-},\texttt{)} ed è una stringa ben formata. Questo implica tra le altre cose che \textbf{$N$ è multiplo di 4}.
	\item Nel caso in cui la successione iniziale di magneti sia già stabile e non ci sia quindi bisogno di girare alcun magnere, rispondere il valore 0.
\end{itemize}

\Scoring
Il tuo programma verrà testato su diversi test case raggruppati in subtask.
Per ottenere il punteggio relativo ad un subtask, è necessario risolvere
correttamente tutti i test relativi ad esso.

\begin{itemize}[nolistsep,itemsep=2mm]
  \item \textbf{\makebox[2cm][l]{Subtask 1} [10 punti]}: Casi d'esempio.
  \item \textbf{\makebox[2cm][l]{Subtask 2} [20 punti]}: $N \leq 20$.
  \item \textbf{\makebox[2cm][l]{Subtask 3} [40 punti]}: $N \leq 1000$.
  \item \textbf{\makebox[2cm][l]{Subtask 4} [30 punti]}: Nessuna limitazione specifica.
\end{itemize}

% Esempi
\Examples
\begin{example}
\exmp{
8
(+-)(-+)
}{%
1
}%
\end{example}
\begin{example}
\exmp{
12
(+-)(+-)(-+)
}{%
1
}%
\end{example}


\Explanation
Nel \textbf{primo caso di esempio} è sufficiente ruotare il primo magnete.\\[2mm]
Nel \textbf{secondo caso di esempio} è sufficiente ruotare l'ultimo magnete.
