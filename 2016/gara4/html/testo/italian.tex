\usepackage{xcolor}
\usepackage{afterpage}
\usepackage{pifont,mdframed}
\usepackage[bottom]{footmisc}

\makeatletter
\gdef\this@inputfilename{input.txt}
\gdef\this@outputfilename{output.txt}
\makeatother

\newcommand{\inputfile}{\texttt{input.txt}}
\newcommand{\outputfile}{\texttt{output.txt}}

\newenvironment{warning}
  {\par\begin{mdframed}[linewidth=2pt,linecolor=gray]%
    \begin{list}{}{\leftmargin=1cm
                   \labelwidth=\leftmargin}\item[\Large\ding{43}]}
  {\end{list}\end{mdframed}\par}

William sta cercando di scrivere un parser HTML facendo uso di espressioni regolari (regexp), così da evocare Zalgo. Per fortuna questo parser non deve davvero essere in grado di consumare HTML, bensì è sufficiente che sia in grado di effettuare il cosidetto ``unescape'': l'operazione inversa dell'escape.

L'escape è una tecnica che viene utilizzata per rappresentare caratteri che altrimenti avrebbero un significato specifico nel linguaggio utilizzato. Per questa particolare applicazione, William ha bisogno di effettuare l'unescape soltanto del carattere \texttt{\&} che, quando viene normalmente sottoposto all'escape in HTML, diventa \texttt{\&amp;} passando quindi da $1$ carattere a $5$ caratteri.

William ha notato però una cosa: una volta completato l'unescape di un intero file HTML, potrebbe essere possibile \emph{eseguire di nuovo l'unescape sullo stesso file}, dato che nel frattempo il codice HTML è cambiato e potrebbero essere comparse altre entità \texttt{\&amp;} da convertire.

Dato il file HTML, calcola il contenuto del file HTML alla fine di questa serie di lanci del parser.

\Implementation
Dovrai sottoporre esattamente un file con estensione \texttt{.c}, \texttt{.cpp} o \texttt{.pas}.

\begin{warning}
Tra gli allegati a questo task troverai un template (\texttt{html.c}, \texttt{html.cpp}, \texttt{html.pas}) con un esempio di implementazione da completare.
\end{warning}

Se sceglierai di utilizzare il template, dovrai implementare la seguente funzione:
\begin{center}\begin{tabularx}{\textwidth}{|c|X|}
\hline
C/C++  & \verb|int unescape(int N, char S[]);|\\
\hline
Pascal & \verb|function unescape(N: longint; var S: array of char): longint;|\\
\hline
\end{tabularx}\end{center}
In cui:
\begin{itemize}[nolistsep]
  \item L'intero $N$ rappresenta la lunghezza (in caratteri) del file HTML.
  \item L'array \texttt{S}, indicizzato da $0$ a $N-1$, contiene i singoli caratteri che compongono il file HTML.
  \item La funzione deve modificare l'array \texttt{S} in modo che rifletta il contenuto finale del file HTML.
  \item La funzione deve inoltre restituire un intero: la lunghezza del file HTML finale.
\end{itemize}

\InputFile
Il file \inputfile{} può contenere diverse righe. La prima riga contiene il numero intero $N$. Dalla riga seguente, si trovano $N$ caratteri che rappresentano il contenuto del file HTML.

\OutputFile
Il file \outputfile{} è composto dallo stesso numero di righe presenti in input. La prima riga contiene il numero di caratteri (minore o uguale a $N$) che compongono il file HTML al termine delle varie conversioni. Dalla seconda riga in poi saranno stampati gli $N$ caratteri che formano effettivamente il file.

\pagebreak
% Assunzioni
\Constraints
\begin{itemize}[nolistsep, itemsep=2mm]
	\item $2 \le N \le 1\,000\,000$.
	\item $S$ è formata da caratteri ASCII stampabili (a parte l'``a capo'') per ogni $i=0\ldots N-1$. Non è detto che siano presenti i tag, o meglio: $S$ potrebbe avere un contenuto \emph{non valido} come HTML.
	\item Il file termina (come sempre) con un carattere ``a capo'', che è però compreso negli $N$ caratteri totali.
\end{itemize}

\Scoring
Il tuo programma verrà testato su diversi test case raggruppati in subtask.
Per ottenere il punteggio relativo ad un subtask, è necessario risolvere
correttamente tutti i test relativi ad esso.

\begin{itemize}[nolistsep,itemsep=2mm]
  \item \textbf{\makebox[2cm][l]{Subtask 1} [10 punti]}: Casi d'esempio.
  \item \textbf{\makebox[2cm][l]{Subtask 2} [20 punti]}: $N \leq 500$.
  \item \textbf{\makebox[2cm][l]{Subtask 3} [40 punti]}: $N \leq 10\,000$.
  \item \textbf{\makebox[2cm][l]{Subtask 4} [30 punti]}: Nessuna limitazione specifica.
\end{itemize}

% Esempi


\Examples
\begin{example}
\exmpfile{html.input0.txt}{html.output0.txt}%
\exmpfile{html.input1.txt}{html.output1.txt}%
\exmpfile{html.input2.txt}{html.output2.txt}%
\end{example}
