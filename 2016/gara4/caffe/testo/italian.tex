\usepackage{xcolor}
\usepackage{afterpage}
\usepackage{pifont,mdframed}
\usepackage[bottom]{footmisc}

\makeatletter
\gdef\this@inputfilename{input.txt}
\gdef\this@outputfilename{output.txt}
\makeatother

\newcommand{\inputfile}{\texttt{input.txt}}
\newcommand{\outputfile}{\texttt{output.txt}}

\newenvironment{warning}
  {\par\begin{mdframed}[linewidth=2pt,linecolor=gray]%
    \begin{list}{}{\leftmargin=1cm
                   \labelwidth=\leftmargin}\item[\Large\ding{43}]}
  {\end{list}\end{mdframed}\par}

	Durante i lunghi ed estenuanti stage di preparazione dei PO (probabili olimpici), gli $N$ tutor ogni tanto recuperano le energie andando a prendersi un caff\`e. Per ottimizzare la loro presenza in aula, ogni tutor fa pausa in un momento diverso del giorno: in particolare, il tutor $i$-esimo fa pausa nell'intervallo di tempo $[A_i, B_i]$ dove i valori $A_i$ e $B_i$ sono tutti distinti tra loro.

	I tutor tra di loro sono sempre molto cordiali: ogni volta che uno di loro raggiunge la macchinetta del caff\`e, offre sempre un giro di caff\`e a tutti i presenti. Questo elevato consumo di caff\`e sta facendo rapidamente finire le scorte a disposizione della macchinetta. Aiuta Monica a programmare i rifornimenti calcolando quanti caff\`e vengono bevuti ogni giorno!

\Implementation
Dovrai sottoporre esattamente un file con estensione \texttt{.c}, \texttt{.cpp} o \texttt{.pas}.

\begin{warning}
Tra gli allegati a questo task troverai un template (\texttt{caffe.c}, \texttt{caffe.cpp}, \texttt{caffe.pas}) con un esempio di implementazione da completare.
\end{warning}

Se sceglierai di utilizzare il template, dovrai implementare la seguente funzione:
\begin{center}\begin{tabularx}{\textwidth}{|c|X|}
\hline
C/C++  & \verb|int pausa(int N, int A[], int B[]);|\\
\hline
Pascal & \verb|function pausa(N: longint; var A, B: array of longint): longint;|\\
\hline
\end{tabularx}\end{center}
In cui:
\begin{itemize}[nolistsep]
  \item L'intero $N$ rappresenta il numero di tutor.
  \item Gli array \texttt{A} e \texttt{B}, indicizzati da $0$ a $N-1$, descrivono gli intervalli $[A_i, B_i]$ di tempo in cui i tutor vanno a prendersi il caff\`e.
  \item La funzione dovrà restituire il numero totale di caff\`e consumati, che verrà stampato sul file di output.
\end{itemize}

\InputFile
Il file \inputfile{} è composto da $N+1$ righe. La prima riga contiene l'unico intero $N$. La successive $N$ righe contengono ciascuna due interi $A_i$, $B_i$ separati da uno spazio.

\OutputFile
Il file \outputfile{} è composto da un'unica riga contenente un unico intero, la risposta a questo problema.

\pagebreak
% Assunzioni
\Constraints
\begin{itemize}[nolistsep, itemsep=2mm]
	\item $1 \le N \le 40\,000$.
	\item $0 \le A_i < B_i \le 1\,000\,000$ per ogni $i=0\ldots N-1$.
	\item I valori $A_i$ e $B_i$ sono tutti differenti tra loro.
\end{itemize}

\Scoring
Il tuo programma verrà testato su diversi test case raggruppati in subtask.
Per ottenere il punteggio relativo ad un subtask, è necessario risolvere
correttamente tutti i test relativi ad esso.

\begin{itemize}[nolistsep,itemsep=2mm]
  \item \textbf{\makebox[2cm][l]{Subtask 1} [10 punti]}: Casi d'esempio.
  \item \textbf{\makebox[2cm][l]{Subtask 2} [20 punti]}: $N \leq 10$.
  \item \textbf{\makebox[2cm][l]{Subtask 3} [40 punti]}: $N \leq 1000$.
  \item \textbf{\makebox[2cm][l]{Subtask 4} [30 punti]}: Nessuna limitazione specifica.
\end{itemize}

% Esempi


\Examples
\begin{example}
\exmpfile{caffe.input0.txt}{caffe.output0.txt}%
\exmpfile{caffe.input1.txt}{caffe.output1.txt}%
\end{example}


\Explanation
Nel \textbf{primo caso di esempio}, prima arriva il tutor 2 che si prende il suo caff\`e, poi il tutor 0 che lo offre al tutor 2 e infine il tutor 1 che lo offre a entrambi gli altri per un totale di $1+2+3=6$ caff\`e.\\[2mm]
Nel \textbf{secondo caso di esempio}, il tutor 1 offre il caff\`e al tutor 0, il tutor 2 lo offre al tutor 1, il tutor 3 lo offre al tutor 5 per un totale di $1 + 2 + 2 + 2 + 1 + 1 = 9$ caff\`e.
