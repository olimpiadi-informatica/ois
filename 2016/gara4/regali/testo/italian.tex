\usepackage{xcolor}
\usepackage{afterpage}
\usepackage{pifont,mdframed}
\usepackage[bottom]{footmisc}

\makeatletter
\gdef\this@inputfilename{input.txt}
\gdef\this@outputfilename{output.txt}
\makeatother

\newcommand{\inputfile}{\texttt{input.txt}}
\newcommand{\outputfile}{\texttt{output.txt}}

\newenvironment{warning}
  {\par\begin{mdframed}[linewidth=2pt,linecolor=gray]%
    \begin{list}{}{\leftmargin=1cm
                   \labelwidth=\leftmargin}\item[\Large\ding{43}]}
  {\end{list}\end{mdframed}\par}

	William deve andare a comprare un regalo di compleanno per i suoi due giovani cuginetti. Essendo fratelli gemelli, sa che se gli facesse due regali troppo diversi sicuramente inizierebbero a litigare. Pertanto si \`e procurato un listino di un negozio contenente la descrizione di $N$ giocattoli, ciascuno individuato da $Q$ qualit\`a (rappresentate con dei numeri da $1$ a $1000$).

	Dato che le qualit\`a dei giocattoli sono elencate in ordine di importanza e il significato di una qualit\`a dipende dalle precedenti, la simiglianza di due giocattoli \`e definita come \emph{il pi\`u lungo prefisso comune} tra le loro due descrizioni. Trova i giocattoli pi\`u simili tra loro!

\Implementation
Dovrai sottoporre esattamente un file con estensione \texttt{.c}, \texttt{.cpp} o \texttt{.pas}.

\begin{warning}
Tra gli allegati a questo task troverai un template (\texttt{regali.c}, \texttt{regali.cpp}, \texttt{regali.pas}) con un esempio di implementazione da completare.
\end{warning}

Se sceglierai di utilizzare il template, dovrai implementare la seguente funzione:
\begin{center}\begin{tabularx}{\textwidth}{|c|X|}
\hline
C/C++  & \verb|int compra(int N, int Q, int* G[]);|\\
\hline
Pascal & \verb|function compra(N, Q: longint; var V: matrix): longint;|\\
\hline
\end{tabularx}\end{center}
In cui:
\begin{itemize}[nolistsep]
  \item L'intero $N$ rappresenta il numero di giocattoli presenti nel catalogo.
  \item L'intero $Q$ rappresenta il numero di qualit\`a che descrivono ogni giocattolo.
  \item L'array bidimensionale \texttt{G}, indicizzato da $0$ a $N-1$ e da $0$ a $Q-1$, contiene le qualit\`a dei giocattoli.
  \item La funzione dovrà restituire la massima simiglianza possibile tra due giocattoli, che verrà stampata sul file di output.
\end{itemize}

\InputFile
Il file \inputfile{} è composto da $N+1$ righe. La prima riga contiene i due interi $N$ e $Q$. La successive $N$ righe contengono ciascuna i $Q$ interi $G_{i,j}$ separati da uno spazio.

\OutputFile
Il file \outputfile{} è composto da un'unica riga contenente un unico intero, la risposta a questo problema.

% Assunzioni
\Constraints
\begin{itemize}[nolistsep, itemsep=2mm]
	\item $2 \le N \le 10\,000$.
	\item $1 \le Q \le 100$.
	\item $1 \le G_{i,j} \le 1000$ per ogni $i=0\ldots N-1$, $j=0\ldots Q-1$.
\end{itemize}

\Scoring
Il tuo programma verrà testato su diversi test case raggruppati in subtask.
Per ottenere il punteggio relativo ad un subtask, è necessario risolvere
correttamente tutti i test relativi ad esso.

\begin{itemize}[nolistsep,itemsep=2mm]
  \item \textbf{\makebox[2cm][l]{Subtask 1} [10 punti]}: Casi d'esempio.
  \item \textbf{\makebox[2cm][l]{Subtask 2} [20 punti]}: $N,Q \leq 10$.
  \item \textbf{\makebox[2cm][l]{Subtask 3} [40 punti]}: $N \leq 100$.
  \item \textbf{\makebox[2cm][l]{Subtask 4} [30 punti]}: Nessuna limitazione specifica.
\end{itemize}

% Esempi


\Examples
\begin{example}
\exmpfile{regali.input0.txt}{regali.output0.txt}%
\exmpfile{regali.input1.txt}{regali.output1.txt}%
\end{example}


\Explanation
Nel \textbf{primo caso di esempio}, ogni coppia di giocattoli ha esattamente le prime quattro qualit\`a in comune.\\[2mm]
Nel \textbf{secondo caso di esempio}, il primo e l'ultimo giocattolo hanno le prime tre qualit\`a in comune e sono i pi\`u simili.
