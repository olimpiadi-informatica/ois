\usepackage{xcolor}
\usepackage{afterpage}
\usepackage{pifont,mdframed}
\usepackage[bottom]{footmisc}

\makeatletter
\gdef\this@inputfilename{input.txt}
\gdef\this@outputfilename{output.txt}
\makeatother

\newcommand{\inputfile}{\texttt{input.txt}}
\newcommand{\outputfile}{\texttt{output.txt}}

\newenvironment{warning}
  {\par\begin{mdframed}[linewidth=2pt,linecolor=gray]%
    \begin{list}{}{\leftmargin=1cm
                   \labelwidth=\leftmargin}\item[\Large\ding{43}]}
  {\end{list}\end{mdframed}\par}

	William has just downloaded the new \emph{Drop-It!} game on his smartphone. In this game several blocks of different lengths are dropped by a crane one on top of the other. While falling, they follow these rules:
	\begin{enumerate}
		\item If a block falls on top of another one of equal length, the fall stops and both blocks involved are destroyed (they disappear);
		\item If a block falls on top of a longer block, the fall stops and an extra block appears on top of the last one, with length equal to the length difference of the two blocks involved;
		\item If a block falls on top of a shorter block $B$, its length is reduced by the length of $B$, $B$ is destroyed and the fall continues.
	\end{enumerate}
	Help William plan his strategy, by writing a program that computes the stack produced by the falling of $N$ blocks of length $L_i$ where $i=0, \ldots, N$.

\Implementation
You shall submit exactly one file having extension \texttt{.c}, \texttt{.cpp} o \texttt{.pas}.

\begin{warning}
Among the attachments of this task you will find a template (\texttt{dropit.c}, \texttt{dropit.cpp}, \texttt{dropit.pas}) with a sample incomplete implementation.
\end{warning}

If you use the template, you'll need to implement the following function:
\begin{center}\begin{tabularx}{\textwidth}{|c|X|}
\hline
C/C++  & \verb|int cadi(int N, int L[], int P[]);|\\
\hline
Pascal & \verb|function cadi(N: longint; var L, P: array of longint): longint;|\\
\hline
\end{tabularx}\end{center}
Where:
\begin{itemize}[nolistsep]
  \item $N$ is the number of blocks falling from the crane.
  \item \texttt{L} is an array indexed from $0$ to $N-1$, containing the lengths of the blocks dropped by the crane (in falling order).
  \item \texttt{P} is an array indexed from $0$ to $M-1$, that must be filled with the lengths of the blocks that form the final stack (in the same order as the input's).
  \item The function shall return $M$, the number of blocks in the final stack. This number will be printed to the output file along with the contents of the \texttt{P} array.
\end{itemize}

\InputFile
File \inputfile{} consists of two lines. Line $1$ contains integer $N$. Line $2$ contains integers $L_i$ separated by spaces.

\OutputFile
File \outputfile{} consists of two lines. Line $1$ contains integer $M$. Line $2$ contains integers $P_i$ separated by spaces.

% Assunzioni
\Constraints
\begin{itemize}[nolistsep, itemsep=2mm]
	\item $1 \le N \le 100\,000$.
	\item $1 \le L_i \le 10\,000$ for all $i=0\ldots N-1$.
\end{itemize}

\Scoring
Your program will be tested against several test cases grouped in subtasks.
In order to obtain a subtask's score, your program needs to correctly solve all of its test cases.

\begin{itemize}[nolistsep,itemsep=2mm]
  \item \textbf{\makebox[2cm][l]{Subtask 1} [10 punti]}: Sample test cases.
  \item \textbf{\makebox[2cm][l]{Subtask 2} [20 punti]}: $N \leq 10$.
  \item \textbf{\makebox[2cm][l]{Subtask 3} [40 punti]}: $N \leq 1000$.
  \item \textbf{\makebox[2cm][l]{Subtask 4} [30 punti]}: No limits.
\end{itemize}

% Esempi


\Examples
\begin{example}
\exmpfile{dropit.input0.txt}{dropit.output0.txt}%
\exmpfile{dropit.input1.txt}{dropit.output1.txt}%
\end{example}


\Explanation
In the \textbf{first sample test case}, rule (1) is applied first, thus clearing the current stack. Then rule (2) is applied, thus producing an extra block of length $22$.\\[2mm]
In the \textbf{second sample test case}, rule (2), (1), (3) and (2) get applied in order, thus obtaining the following succession of states:
\begin{center}
\texttt{17 <---- 13\\
17 13 4 <---- 4\\
17 13  <---- 15\\
17 <---- 2\\
17 2 15
}
\end{center}
